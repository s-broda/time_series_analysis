\documentclass[11pt, a4paper]{article}
\usepackage{amsmath, amsfonts, dsfont, booktabs, graphicx, natbib, a4wide, times, microtype}
\begin{document}
\title{Exercise 1}
\author{Simon A.\ Broda}
\date{}
\maketitle

\begin{enumerate}
\item
\begin{enumerate}
\item \label{q1} Open the file \texttt{maunaloa.wf1}; this is a famous data set used in machine learning. Make a time series plot.
\item \label{q2}Estimate a linear trend by regressing the \texttt{co2} series on an intercept and the variable \texttt{time}.
\item \label{q3}Plot the data, together with the estimated linear trend.
\item \label{q4} Produce a forecast for 2005M1, first manually using the fitted model
\[
\widehat{Y_t} = \widehat{\beta_0}+\widehat{\beta_1} t,
\]
then using Eviews.
\item Repeat Questions \ref{q2} through \ref{q4}, but using a quadratic trend.
\item Repeat Questions \ref{q2} through \ref{q4}, but using an exponential trend.
\end{enumerate}
\item 
\begin{enumerate}
\item Compute the 3rd order moving average of the \texttt{co2} series for 1964M6 by hand.
\item Estimate the trend with a 12 month moving average (12 months are necessary to cover a full cycle). Then plot the resulting trend estimate and the data together in a time series plot.
\end{enumerate}
\item 
\begin{enumerate}
\item Estimate a model with a linear trend and 12 monthly dummies (and no intercept) for the \texttt{co2} series. Then, produce an (in-sample) forecast for 2004M12, both by hand and using EViews. Also create an actual-fitted-residual plot.
\item Same, but include an intercept and remove the last dummy.
\end{enumerate}
\end{enumerate}
\end{document} 